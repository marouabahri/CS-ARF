\section{Introduction}
\label{Introduction}

%In order to learn with data whose nature changes over time, 
%we propose %to study and develop 
%several strategies. 
 %This current work constitutes our 
 Our contributions in this area of research %, wich 
 can be divided in three main ones: a new unified way to look at the problem of concept drift, the proposal of 
two estimators algorithms that detect change using a sliding 
window {\tt ADWIN} and {\tt K-ADWIN} (combination of {\tt ADWIN} with Kalman filters),
and the application to two learners algorithms: Na\"{\i}ve Bayes and K-means.

\input{ConIntrod1}
%an algorithm {\tt ADWIN} to detect change using a sliding window, and the combination of {\tt ADWIN} with Kalman filters.

\BEGINOMIT
is one of the core problems in data mining and machine learning.
To mine or learn such data, one needs strategies for
the following three tasks, at least: 1) detecting when
change occurs 2) deciding which examples to keep and which ones
to forget (or, more in general, keeping updated sufficient statistics),
and 3) revising the current model(s) when significant  
change has been detected.

Most strategies use variations of the {\em sliding window} idea:
a window is maintained that keeps the most recently read examples,
and from which older examples are dropped according to some
set of rules. The contents of the window can be used for the
three tasks above: 1) to detect change (e.g., by using some statistical
test on different subwindows), 2) obviously, to
obtain updated statistics from the recent examples,
and 3) to have data to rebuild or revise the model(s) after data has 
changed.

The simplest rule is to keep a window
of some fixed size, usually determined {\em a priori} by the user.
This can work well if information on the time-scale
of change is available, but this is rarely the case.
Normally, the user is caught in a tradeoff without solution:
choosing a small size (so that the window reflects accurately the current distribution)
and choosing a large size (so that many examples are available to work on, 
increasing accuracy in periods of stability).
A different strategy uses a {\em decay function}
to weight the importance of examples according to their
age (see e.g. \cite{CS03}). %strauss
In this case, the tradeoff shows up in the 
choice of a decay constant that should match the unknown rate of change.

Less often, it has been proposed to use windows of variable size.
In general, one tries to keep examples as long as possible, i.e., 
while not proven stale. This delivers
the users from having to guess {\em a priori} an unknown parameter such
as the time scale of change. However, most works along these lines 
that we know of (e.g., \cite{Gama,Klinkenberg,Last,WidmerKubat})
are heuristics and have no rigorous guarantees of performance. 
Some works in computational learning theory 
(e.g. \cite{bartlett00,helmbold94tracking,herbster95tracking}) 
describe strategies with rigorous performance
bounds, but to our knowledge they have never been tried
in real learning/mining contexts and often assume a known bound 
on the rate of change. 

In addition, window strategies have been used in conjunction
with learning/mining algorithms in two ways: one,
externally to the learning algorithm; the window is used
to monitor the error rate of the current model, which under
stable distributions should keep decreasing or at most stabilize;
when instead this rate grows significantly, change is declared and
the base learning algorithm is invoked to revise or rebuild the
model with fresh data. Note that in this case the window
contains bits or real numbers (not full examples).
The other way is to embed the window system {\em inside} the learning
algorithm, to maintain the statistics required by the learning
algorithm continuously updated; it is then the algorithm's responsibility
to keep the model in synchrony with these statistics. 
\ENDOMIT
%In this paper, 
Our first work \cite{bif-gav}
proposes a new algorithm ({\tt ADWIN}, for ADaptive
WINdowing) %\cite{bif-gav}
for maintaining a window of variable size containing bits or real numbers.
The algorithm automatically grows the window when no change is apparent,
and shrinks it when data changes.
Unlike many related works, we provide rigorous guarantees of
its performance, in the form of bounds on the rates of false positives
and false negatives.
In fact, it is possible to show that for some change structures, {\tt ADWIN}
automatically adjusts its window size to the optimum balance point
between reaction time and small variance.
Since {\tt ADWIN} keeps bits or real numbers, it can be put to work
together with a learning algorithm in the first way, that is,
to monitor the error rate of the current model.

The first version of {\tt ADWIN} is inefficient in time and memory.
Using ideas from data-stream algorithmics,
we provide another version, {\tt ADWIN2}, working in low memory and
time. In particular, {\tt ADWIN2} keeps a window of length $W$ with
$O(\log W)$ memory and update time, while keeping essentially
the same performance guarantees as {\tt ADWIN} (in fact, it does
slightly better in experiments).
Because of this low time and memory requirements, it is thus possible
to use {\tt ADWIN2} in the second way: a learning algorithm
can create many instances of {\tt ADWIN2} to maintain updated
the statistics (counts, averages, entropies, \dots) from which it
builds the model. 


To test our approach, we perform two types of experiments.
In the first type, we test the ability of {\tt ADWIN2}
to track some unknown quantity, independent of any learning.
We generate a sequence of random bits with some hidden
probability $p$ that changes over time. We check the rate
of false positives (\% of claimed changes when $p$ does not
really change) and false negatives (\% of changes missed
when $p$ does change) and in this case the time until
the change is declared. We compare {\tt ADWIN2} with
a number of fixed-size windows and show, as expected,
that it performs about as well or only slightly worse than the best
window for each rate of change, and performs far better than
each windows of any fixed-size $W$ when the change of rate is
very different from $W$. We also compare
to one of the recently proposed variable-size window methods \cite{Gama}
and show that it performs better, for moderately large quantities of data.

Then we test {\tt ADWIN2} in conjunction with two %a 
learning algorithms.
In this first work, we choose the Na\"\i ve Bayes (NB) predictor  and a $k$-means clusterer since
it is easiest to observe their reactions % its reaction
 to time changes. %In the long version of the paper we report on experiments with 
%a $k$-means clusterer.
%We are currently working on the application to decision tree induction. 
We try both using {\tt ADWIN2} ``outside'', monitoring NB's
error rate, and ``inside'', providing accurate statistics to NB.
We compare them to fixed-size windows and the 
 variable-length window strategy in \cite{Gama}.
We perform experiments both on synthetic and real-life data.
The second combination ({\tt ADWIN2} inside NB) performs best, sometimes
spectacularly so. The first combination performs about as well
as \cite{Gama} in some cases, and substantially better in others.

\BEGINOMIT
{\bf NOTE:} Due to space limitations, several discussions,
technical details, and results of experiments are omitted in this 
version. They can be found in the long version, available from the authors'
homepages.
\ENDOMIT

Our second work \cite{Kbif-gav} proposes the combination of a classical estimation method
in automatic control theory, the Kalman filter, with 
{\tt ADWIN} as an algorithm for adaptively changing the size of the window in reaction to changes observed in the data. 

%One of the most widely used estimation algorithms is the Kalman filter, an algorithm that generates 
%estimates of variables of the system being controlled by processing available sensor measurements. 
Kalman filtering and related estimation algorithms
have proved tremendously useful in a large variety of settings. 
Automatic machine learning is but one of them; 
see  \cite{gama-apneas,jacob-04} among many others. 
There is however an important difference in the 
control theory and machine learning settings: 

In automatic control, we assume that system parameters are known or easily detectable; 
these parameters are physical properties of devices, and therefore fixed. 
In contrast, in most machine learning situations the distribution that generates the examples
is totally unknown, and there is no obvious way to measure any of its statistics, 
other than estimating them from the data. In addition, these statistics 
may vary impredictably over time, either continuously at a slow rate, or abruptly from time to time. 

We combine {\tt ADWIN} and Kalman filter and
compare experimentally the performance of the resulting algorithm, 
{\tt K-ADWIN}, with other estimator algorithms. %The intuition
%why this combination should be better than {\tt ADWIN} alone
%or the Kalman filter alone is as follows. 
\BEGINOMIT
The Kalman filter is a memoryless algorithm, and it can benefit from having
a memory aside. In particular, running a Kalman filter requires
knowledge of at least two parameters of the system, named
{\em state covariance} and {\em measurement covariance}, that should
be estimated a priori. These are generally difficult to measure in the context
of learning from a data stream, and in addition they can vary over time. 
The window that {\tt ADWIN} maintains adaptively is guaranteed
to contain up-to-date examples from which the current value of
these covariances can be estimated and used in the Kalman filter. 

On the other hand, {\tt ADWIN} is somewhat slow in detecting
a gradual change, because it gives the same weight
to all examples in the window -- it is what we will call a {\em linear}
estimator. If there is a slow gradual change, 
the most recent examples should be given larger weight. This is
precisely what the Kalman filter does in its estimation. 
\ENDOMIT
%As in \cite{bif-gav}, %section \ref{ch:conexperiments}
As with {\tt ADWIN}, we test {\tt K-ADWIN} on two well-known learning 
algorithms where it is easy to observe the effect of distribution drift:
the Na\"\i ve Bayes classifier and the $k$-means
clusterer.
\BEGINOMIT
 We also perform experiments that directly compare the ability
of different estimators to track the average value of a stream of real numbers
that varies over time. 
We use synthetic data in order to control precisely
the type and amount of distribution drift. The main conclusions are: 

\begin{itemize}
\item In all three types of experiments (tracking, Na\"\i ve Bayes, and $k$-means), 
{\tt K-ADWIN} either gives best results or is very close in performance to the best 
of the estimators we try. And each of the other estimators is 
clearly outperformed by {\tt K-ADWIN}
in at least some of the experiments. In other words, no estimator ever does
much better than {\tt K-ADWIN}, and each of the others 
is outperformed by {\tt K-ADWIN} in at least one context. 
\item More precisely, for the tracking problem, {\tt K-ADWIN} and {\tt ADWIN} automatically
do about as well as the Kalman filter with the best set of fixed covariance parameters
(parameters which, in general, can only be determined after a good number of experiments). 
And these three do far better than any fixed-size window. 
\item In the Na\"\i ve Bayes experiments, {\tt K-ADWIN} does somewhat better than
{\tt ADWIN} and far better than any memoryless Kalman filter. This is, then, 
a situation where having a memory clearly helps. 
\item In the $k$-means case, again {\tt K-ADWIN} performs about as well 
as the best (and difficult to find) Kalman filter, 
and they both do much better than fixed-size windows.
\end{itemize}
\ENDOMIT

%We now shortly summarize and describe just the main contributions of these current works.
{\change
The next sections in this Chapter basically elaborate on this Introduction. They are necessarily more technical, and could maybe be skipped on a first reading. They present the material in the works~\cite{Kbif-gav,bif-gav} %[ds06,sdm07]
although in a different ordering to highlight the motivation and ideas common to both.
}

